\section{The Heat Equation}
Aytaylik $\Omega\subset \rd$ chegaralangan ochiq to`plam bo`lsin va $\partial\Omega$ Lipshitz uzliksiz. $C_0(\Omega)$ bilan $\overline{\Omega}$ da uzliksiz va $\partial\Omega$ to`plamda aynan nolga teng bo`lgan funksiyalar sinfini belgilab olamiz. Albatta $C_0(\Omega)\subset L^2(\Omega)$ bo`ladi. Ushbu sinfni sup-norma bilan birgalikda qarasak Banah fazosiga aylanadi va bunda $C_0(\Omega)\hookrightarrow L^2(\Omega)$ uzliksiz \textcolor{red}{akslantirish}.
\subsection{The Laplacian on different domains}
\begin{align}
    &\left\{\begin{aligned}
        D(B)&=\left\{u\in H_0^1(\Omega): \Delta u\in L^2(\Omega)\right\}\\
        Bu&=\Delta u, \quad \forall u\in D(B)
    \end{aligned}\right. \label[equation]{Bdefinition}\\
    &\left\{\begin{aligned}
        D(A)&=\left\{u\in H_0^1(\Omega)\cap C_0(\Omega): \Delta u\in C_0(\Omega)\right\}\\
        Au&=\Delta u, \quad \forall u\in D(A)
    \end{aligned}\right. \label[equation]{Adefinition}
\end{align}
\begin{proposition}
    \eqref{Bdefinition} va \eqref{Adefinition} da aniqlangan $A$ va $B$ operatorlar quyidagi hossalarga ega.
    \begin{enumerate}
        \item $D(B)$ to`plam $L^2(\Omega)$ fazoda zich;
        \item $B:D(B)\mapsto L^2(\Omega)$ o`z-o`ziga qo`shma operator;
        \item $B\le 0$ ya`ni $B$ \textcolor{red}{dissipative} operator;
        \item $D(A)$ to`plam $C_0(\Omega)$ fazoda zich;
        \item $A:D(A)\mapsto C_0(\Omega)$ \textcolor{red}{m-dissipative} operator;
    \end{enumerate}
\end{proposition}
\begin{proof}
    
\end{proof}

Lumer-Phillips teoremasiga ko`ra $A$ va $B$ operatorlarning har biri \textcolor{red}{siquvchi yarim-gruppalar generatori} bo`ladi. Biz bu yarim-grupplarni mos ravishda $\left\{T(t)\right\}_{t\ge 0}$ va $\left\{S(t)\right\}_{t\ge 0}$ bilan belgilab olamiz.

\begin{proposition}
    Ixtiyriy $\varphi\in C_0(\Omega)$ va $t\ge 0$ uchun $T(t)\varphi=S(t)\varphi$ tenglik o`rinli bo`ladi.
\end{proposition}

\subsection{The Semilinear Heat Equation}
Aytaylik $F\in C\left(\r,\r\right)$ lokal Lipshitz uzliksiz funksiya va $F(0)=0$ bo`lsin. Biz $F\colon C_0(\Omega)\mapsto C_0(\Omega)$ akslantirishni quyidagicha aniqlaymiz:
\begin{equation*}
    F(u)(x)=F(u(x)),\quad \forall u\in C_0(\Omega), \; \forall x\in\Omega.
\end{equation*}
Bundan kelib chiqadiki $F\colon C_0(\Omega)\mapsto C_0(\Omega)$ ham lokal Lipshitz uz\-lik\-siz aks\-lan\-ti\-rish bo`ladi. 

Berilgan $\varphi\in C_0(\Omega)$ funksiyalar uchun, biz shunday $T>0$ va quyidagi masalani qanoatlantiradigan $u\colon[0,T]\mapsto C_0(\Omega)$ funksiyani qidiramiz:

\begin{empheq}[left=\empheqbiglbrace]{align}
    &u\in C\left([0,T],C_0(\Omega)\right)\cap C\left((0,T], H_0^1(\Omega)\right)\cap C^1\left((0,T], L^2(\Omega)\right);\label[equation]{classforsemilinearheat}\\
    &\Delta u\in C\left((0,T], L^2(\Omega)\right);\label[equation]{laplacianheatsemilinear}\\
    &u_t-\Delta u=F\left(u\right), \quad \forall t\in[0,T];\label[equation]{abstractsemilinearheat}\\
    &u(0)=\varphi.\label[equation]{initaldatasemilinearheat}
\end{empheq}

Quyidagi teoremada biz \eqref{classforsemilinearheat}---\eqref{initaldatasemilinearheat} masalaning yechimi va unga mos integral teng\-la\-ma\-ning yechimi aynan bir xil bo`lishini ko`ramiz.

\begin{theorem}
    Aytaylik $\varphi \in C_0(\Omega)$, $T>0$ va $u\in C\left([0,T], C_0(\Omega)\right)$ bo`lsin. Bu holatda $u\colon [0,T]\mapsto C_0(\Omega)$ funksiya \eqref{classforsemilinearheat}---\eqref{initaldatasemilinearheat} masalaning yechimi bo`lishi uchun 
    \begin{equation}\label[equation]{semilinearheatintegralform}
        u(t)=T(t)\varphi +\int_{0}^{t} T(t-s)F\left(u(s)\right)\,ds,\quad \forall t\in [0,T]
    \end{equation}
    tenglik o`rinli bo`lishi zarur va yetarlidir.
\end{theorem}

\begin{proof}
    
\end{proof}

Biz bilamizki \eqref{semilinearheatintegralform} tenglama lokal yechimga ega va agar ushbu yechimning normasi chegaralangan bo`lsa bu yechim global bo`lishga majbur. Shu faktdan foydalangan holda biz quyida bir nechta holatlarni o`rganamiz.

Avvalo biz quyidagi maksimum prinsipini isbotalab olamiz.
\begin{theorem}[Maksimum prinsipi]
    $T>0$, $f\in C\left([0,T], C_0(\Omega)\right)$ va $\varphi\in C_0(\Omega)$ bo`lsin. Aytaylik $u \in C\left([0,T], C_0(\Omega)\right)\cap C\left((0,T), H_0^1(\Omega)\right)\cap C^1\left((0,T), L^2(\Omega)\right)$ va \hfill\break $\Delta u\in C\left((0,T), L^2(\Omega)\right)$ fuksiya uchun 
    \begin{empheq}[left=\empheqbiglbrace]{align}
        &u'(t)-\Delta u(t)= f(t), \quad \forall t\in (0,T);\\
        &u(0)=\varphi.
    \end{empheq}
\end{theorem}

