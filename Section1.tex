\section{Theory of Distributions}

\begin{center}
    \Large Taqsimotlar nazariyasi \\
    (yohud umumlashgan funksiyalar nazariyasi)
\end{center}

\subsection{Space of test functions}
\textbf{Test funksiyalar fazolari}

Ushbu bo`limda biz ixtiyoriy $\Omega\subseteq \rd$ ochiq to`plam uchun bir nechta funksiyanal fazolarni qaraymiz. $\Omega=\rd$ bo`lgan hollarda biz ayrim belgilashlarda shunchaki $\rd$ tushurib ketamiz. 

\begin{definition}
    $f:\Omega \rightarrow \c $ funksiyaning (\textcolor{red}{supporti}) \emph{dastagi} deb
    \begin{align}
        \left\{x\in \Omega: f(x)=0  \right\}
    \end{align}
    to`plamning $\rd$ fazodagi \textcolor{red}{yopilmasiga} aytiladi va $\supp(f)$ bilan belgilanadi. 

    Agar $\supp(f)$ kompakt to`plam bo`lsa, $f$ funkisya \emph{kompakt dastakli} deb ataladi.
\end{definition}

Istalgan $n\ge 0$ butun son uchun $\Omega\subseteq\rd$ to`plamda quyidagi funksiyalar sinflarini qaraymiz.

\begin{enumerate}[label=\textnormal(\emph{\roman*}\textnormal)]
    \item $\mathscr{E}^n(\Omega)=\left\{f:\Omega\rightarrow \c\,  \left|\right. f\in C^n(\Omega)\right\}$ ya'ni $k$ marta uzliksiz differensiallanuvchi funk\-si\-yalar to`plami. Agar $n=0$ bo`lsa, $\mathscr{E}^0(\Omega)$ bilan $\Omega$ to`plamda aniqlangan uzliksiz funkiyalar to`plamini belgilab olamiz.
    \item $\mathscr{E}(\Omega)$ bilan $\Omega$ to`plamda aniqlangan \emph{silliq} funksiyalar sinfini belgilab olamiz, ya'ni:
    \begin{align}
        \mathscr{E}(\Omega)=\bigcap\limits_{n=0}^{\infty}\mathscr{E}^n(\Omega)
    \end{align}
    \item Aylaylik $K\subset \Omega\subseteq \rd$ kompakt to`plam, $\mathscr{D}_K^n(\Omega)$ va $\mathscr{D}_K(\Omega)$ sinflarni quyidagicha aniq\-lay\-miz
    \begin{align}
        \mathscr{D}_K^n(\Omega)&=\left\{f\in\mathscr{E}^k(\Omega)\;  \left| \;\; \supp(f)\subseteq K \right. \right\}\\
        \mathscr{D}_K(\Omega)&=\left\{f\in\mathscr{E}(\Omega)\;  \left| \;\; \supp(f)\subseteq K \right. \right\}=\bigcap\limits_{n=0}^{\infty}\mathscr{D}_K^n(\Omega)
    \end{align}
    \item $\mathscr{K}(\Omega)$ esa $\Omega$ to`plamning barcha kompakt qism-to`plamlari sinfini belgilasin. $\mathscr{D}(\Omega)$ bilan $\Omega$ to`plamda aniqlangan kompakt dastakli silliq funsiyalar sinfini belgilaymiz.
    \begin{equation}
        \mathscr{D}(\Omega)=\bigcup\limits_{K\in\mathscr{K}(\Omega)}\mathscr{D}_K(\Omega)
    \end{equation}
\end{enumerate}
Ha dastlab bu funksiyalar sinflari juda abstrakt va juda kichik sinflar bo`lib tuyilishi mumkin. Lekin aslida bu sinflar biz bilgan $L^{p}(\Omega)$ fazolarda zich va bunday funksiyalar yordamida biz juda ko`p hossalarni isbotlashimiz mumkin. 

\begin{example}
    \begin{enumerate}
        \item $d=1$ va $\Omega=\r$ bo`lgan holatni qaraylik. Quyidagicha aniq\-lan\-gan $g:\r\rightarrow \r$ funksiyani qaraylik:
    \begin{equation}
            g(x)=\left\{\begin{aligned}
                \exp\left(-\frac{1}{1-\lvert x\rvert^2}\right)&, & \textnormal{ agar } & \lvert x\rvert <1 \\
                 0 &, & \textnormal{ agar } & \lvert x\rvert \ge 1.
            \end{aligned}\right. 
    \end{equation}

    To`g`ri birinchi bor bunday funksiyani tasavvur qilish, va uni nima sa\-bab\-dan biz bu funksiyani qarashimiz kerakligini anglab olish qiyin lekin ozgina fikr yu\-ri\-tib ushbu funksiya qanday hossalarga ega ekanligini ko`rish mumkin. Keling shu yer\-da ushbu funksiya grafigini qaraylik.

    \noindent
    \begin{minipage}{0.9\textwidth}
    \begin{minipage}[b]{0.5\textwidth}
    O`ng tomondagi grafikdan shuni ko`\-ri\-shi\-miz mumkinki $g$ kom\-pakt das\-tak\-li funksiya va $\supp(g)=[-1,1]$. Hosila ta'rifidan foydalanib $g$ funk\-si\-ya $-1$ va $1$ nuq\-ta\-lar\-da ham cheksiz ko`p ma\-ro\-ta\-ba dif\-fe\-ren\-si\-al\-la\-nuv\-chi bo`\-li\-shi\-ni ko`rish mumkin. Bundan ko`rinadiki 
    \begin{align*} 
        g\in \mathscr{D}(\r)\subset\mathscr{E}(\r)
    \end{align*}
    \end{minipage}
    \hspace{0.02\textwidth}
    \begin{minipage}[b]{0.45\textwidth}
        \begin{figure}[H]
            \centering
            \begin{tikzpicture}
            \pgfplotsset{width=\textwidth, height=5cm}
            \begin{axis}[clip=false,
            axis lines=middle,
            xmin=-2, xmax=2,
            ymin=-0.65, ymax=0.65,
            xtick={0, 1,-1,-0.5,0.5},    
            xticklabels={$0$, $1$, $-1$, $-\frac{1}{2}$, $\frac{1}{2}$},
            ytick={0.5,-0.5},
            yticklabels={$\frac{1}{2}$, $-\frac{1}{2}$},
            xticklabel style={anchor=north, font=\scriptsize},
            yticklabel style={font=\small},
            xlabel=$x$, ylabel=$g$, xlabel style={anchor=west}, ylabel style={anchor=south},
            xmajorgrids=true,
            grid style=dashed,
            legend entries={$g$}
            ]
            \addplot[color=blue,domain=-0.9999:0.9999]{exp((-1)/(1-x*x))};
            \addplot[color=blue,domain=-2:-0.9999]{0};
            \addplot[color=blue,domain=0.9999:2]{0};
            \end{axis}
        \end{tikzpicture}
            \label{fig:convkernel}
        
        \end{figure}
    \end{minipage}
    \end{minipage}
\break
Yuqorida aniqlangan $g$ funksiya yordamida biz juda ko`p kompakt dastakli sil\-liq funksiyalar hosil qila olamiz. Masalan, $f(x)=\sin(x)\cdot g(x)\in\mathscr{D}(\r)$. Yoki istalgan silliq funksiyani $g(x)$ ga ko`paytirish orqali biz $\mathscr{D}(\r)$ sinfga tegishli funksiya hosil qilamiz.

\item Aytaylik $d\ge 1$ ixtiyoriy butun son bo`lsin. Yuqoridagi $g$ funksiyaning $\rd$ fazoda ham \textcolor{red}{analogi} mavjud va uni soddagina 
\begin{equation*}
    \rd \ni x\longmapsto g\left(\lVert x\rVert\right)
\end{equation*}
kabi aniqlashimiz mumkin. Bu funksiyaning dastagi ese $\overline{B_1(0)}$---birlik yopiq shar bo`ladi.

\item (\emph{Gaussian}) $g_\alpha : \rd \mapsto \r$ $(d\ge1)$ quyidagicha aniqlangan bo`lsin 
\begin{equation*}
    g_\alpha(x)=e^{-\alpha \lvert x\rvert^2}, \quad (\alpha>0).
\end{equation*}
Albatta bu funksiya kompakt dastakli emas, lekin $g_\alpha \in\mathscr{E}(\rd)$. Keyinchalik biz bu funksiyalar Furye transform bilan bog`liq ajoyib hossalarga ega ekanligini ko`\-ri\-shi\-miz mumkin. Aytish lozim bo`lgan yana bir jihati shunda-ki, $g_\alpha$ va uning barcha hosilalari $\lvert x\rvert \to \infty$  bo`lganda, ixtiyoriy ratsional funksiyaga nisbatan kuchliroq $0$ ga yaqinlashadi. Buni quyidagicha ifodalash mumkin,
\begin{equation*}
    \sup\limits_{x\in\rd} \lvert x^\beta \partial^\gamma g_\alpha(x)\rvert \le C_{\beta,\gamma} \footnotemark
\end{equation*}
\footnotetext{$\beta=(\beta_1,\beta_2,\ldots,\beta_d) \in \mathbb{N}_0^d$ bo`lsa $x^\beta=x_1^{\beta_1}\cdots  x_d^{\beta_d}$ va $\partial^\beta=\frac{\partial^{\beta_1+\beta_2+\ldots+\beta_d}}{\partial_{x_1}^{\beta_1}\partial_{x_2}^{\beta_2}\cdots \partial_{x_d}^{\beta_d}}$}
ixtiyoriy $\beta, \gamma\in \mathbb{N}_0^d$, multiindekslar uchun. 
\end{enumerate}
\end{example}

 