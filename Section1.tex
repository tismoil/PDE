\section{Theory of Distributions}

\begin{center}
    \Large Taqsimotlar nazariyasi \\
    (yohud umumlashgan funksiyalar nazariyasi)
\end{center}

\subsection{Spcae of test functions}
\textbf{Test funksiyalar fazolari}

Ushbu bo`limda biz ixtiyoriy $\Omega\subseteq \rn$ ochiq to`plam uchun bir nechta funksiyanal fazolarni qaraymiz. $\Omega=\rn$ bo`lgan hollarda biz ayrim belgilashlarda shunchaki $\rn$ tushurib ketamiz. 

\begin{definition}
    $f:\Omega \rightarrow \c $ funksiyaning \colorbox{green}{supporti} \textit{dastagi} deb
    \begin{align}
        \left\{x\in \Omega: f(x)=0  \right\}
    \end{align}
    to`plamning $\rn$ fazodagi \colorbox{green}{yopilmasiga} aytiladi va $\operatorname{supp}f$ bilan belgilanadi. 

    Agar $\operatorname{supp}f$ kompakt to`plam bo`lsa, $f$ funkisya \textit{kompakt dastakli} deb ataladi.
\end{definition}

Istalgan $k\ge 0$ butun son uchun $\Omega\subseteq\rn$ to`plamda quyidagi funksiyalar sinflarini qaraymiz.

\begin{enumerate}
    \item $\mathscr{E}^k(\Omega)=\left\{f:\Omega\rightarrow \c\,  \left|\right. f\in C^k(\Omega)\right\}$ ya'ni $k$ marta uzliksiz differensiallanuvchi funk\-si\-yalar to`plami. Agar $k=0$ bo`lsa, $\mathscr{E}^0(\Omega)$ shunchaki uzliksiz funkiyalar to`plamini anglatadi.
    \begin{align}
        \mathscr{E}^\infty(\Omega)=\bigcap\limits_{k=0}^{\infty}\mathscr{E}^k(\Omega)
    \end{align}
\end{enumerate}